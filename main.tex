\documentstyle[11pt,apsab]{article}
% * <bcwelliver@lbl.gov> 2018-06-28T20:17:50.799Z:
%
% ^.
\nofiles
\MeetingID{APS18}
%\DateSubmitted{20170630}
\LogNumber{DNP18-2018-TEST}
\SubmittingMemberSurname{Welliver}
\SubmittingMemberGivenName{Bradford}
%\SubmittingMemberID{61020206,USA}
\SubmittingMemberEmail{bcwelliver@lbl.gov}
\SubmittingMemberAffil{Lawrence Berkeley Natl Lab}
\PresentationType{oral}
\SortCategory{?.}{}{}{}
\received{1313 SMarch 1313}
\begin{document}
\Title{Application of Cryogenic TES based Light Detectors for CUPID}
\AuthorSurname{Welliver}
\AuthorGivenName{Bradford}
%\AuthorEmail{bcwelliver@lbl.gov}
\AuthorAffil{Lawrence Berkeley Natl Lab}
\CategoryType{E}
\begin{abstract}
The Cryogenic Underground Observatory for Rare Events (CUORE) is a search for lepton number violating new physics currently operating at the Laboratori Nazionali del Gran Sasso (LNGS). CUORE monitors 988 TeO$_2$ crystals (742 kg) for neutrinoless double beta decay ($0\nu\beta\beta$) by operating these crystals as cryogenic bolometers using neutron-transmutation doped (NTD) Ge thermistors. CUORE is expected to achieve a sensitivity to the $^{130}$Te $0\nu\beta\beta$ decay half-life of $T_{1/2} = 9$ x $10^{25}$ years (90 \% C.L.) after 5 years of operation, and has already met the expected background goals of approximately 1 $\frac{cnt}{keV*kg*yr}$ In order to further improve upon the background the CUORE Upgrade with Particle ID (CUPID) program will introduce improved radiopurity screening, enhanced target masses, and use a two channel energy collection approach (light and heat). This will allow for event by event discrimination of $\alpha$ and $\beta$ events, enhancing the ability to reject background. In this talk I will discuss how the current R\&D at LBNL and UC Berkeley with low-Tc transition edge sensors (TES) with SQUID based light detectors presents a suitable technology to meet CUPID design goals, and how such devices might be realized in the CUPID experiment.


\end{abstract}
\end{document}